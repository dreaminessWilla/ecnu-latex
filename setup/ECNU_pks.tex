%!TEX TS-program = xelatex  
%!TEX encoding = UTF-8 Unicode

%%%%%%%%%%%%%%%%%%%%%%%%%%%%%%%%%%%%%%%%%%%%%%%%%%%%%%%%%%%%%%%%
%                   加载必要的宏包                             %
%%%%%%%%%%%%%%%%%%%%%%%%%%%%%%%%%%%%%%%%%%%%%%%%%%%%%%%%%%%%%%%%
\usepackage{xltxtra}
\setmainfont[BoldFont=黑体]{宋体}

\usepackage{amsmath,amsthm,amssymb,latexsym,xcolor}   % 使用AMS数学公式符号字体定理,页眉,扩展的color
\newcommand\hmmax{0} % default 3
\newcommand\bmmax{0} % default 4
\usepackage{bm}                                                % 希腊字母黑体显示
\usepackage{epsfig,graphicx,picins,picinpar,subfigure,rotating}                  % 使用图形
\usepackage{fancybox,fancyvrb,shortvrb,fancyhdr}                                     % 支持抄录 --- 仅用于生成源代码
\usepackage{cite}  % natbib不能再用了:\usepackage[numbers,sort&compress]{natbib}
% %\usepackage{undertilde}
\usepackage{booktabs}      % 让你的表格中使用不同粗细的横线来划分行

\usepackage{flafter}            % 因为图形可浮动到当前页的顶部,所以它可能会出现
                                % 在它所在文本的前面. 要防止这种情况,可使用 flafter
                                % 宏包
\usepackage[below]{placeins}    %浮动图形控制宏包
                                %允许上一个section的浮动图形出现在下一个
                                %section的开始部分该宏包提供处理浮动对象
                                %的 \FloatBarrier 命令,使所有未处理的浮动
                                %图形立即被处理
\usepackage{array,tabularx,longtable,booktabs,threeparttable,colortbl,multirow,bigstrut}
\usepackage{dcolumn}            % 让表格中将小数点对齐
                                % 或 \hline\hline,当然在和垂直线的交叉处会有所不同。
%---%\usepackage{slashbox}           % 可在表格的单元格中画上一斜线。


%============================版面控制宏包=================================%
% 页边距:上:3.0cm,下:2.0cm,左:2.8cm,右:2.2cm,页眉:2.2cm,页脚1.5cm;
\usepackage[top=1.8cm,bottom=2cm,left=2.8cm,right=2.2cm,includehead,includefoot]{geometry}


%============================页眉页脚控制=================================%
\usepackage[symbol,perpage]{footmisc}  % 脚注控制可使得每页的脚注编码重新复位,
                                       % 但可能导致脚注的链接不正确
                                       % 注意大小写
% %\usepackage{pageno}                    % 章首页的页眉处理, 可以改为自己想要的形式

%======================== 数学公式相关宏包 ===============================%
\usepackage{mathtools}
\usepackage{mathrsfs}               % 不同于\mathcal or \mathfrak 之类的英文花体字体
\usepackage{subeqnarray}            %多个子方程(1-1a)(1-1b)
%\iffalse 以下是一个例子
%\begin{subeqnarray}
%\label{eqw} \slabel{eq0}
% x & = & a \times b \\
%\slabel{eq1}
% & = & z + t\\
%\slabel{eq2}
% & = & z + t
%\end{subeqnarray}
%\fi

%=============================标题与列表宏包=============================%
%\usepackage[sf]{titlesec}           % 控制标题的宏包,配合命令在后面,
                                    % 将cjk+miktex+scrbook+gb.cap下的章的标题号,
                                    % 比如~``第二章 XXX''位置于中心
\usepackage{enumerate}              % 改变列表标号样式宏包 其后可接选项[a,A,i,I,1]
\usepackage{caption2}               % 浮动图形和表格标题样式,可选项为
                                    % [scriptsize,footnotesize,centerlast]
\usepackage{setspace}               % 图形和表格的标题如果是多行,行距比较大,可以加宏包

%\usepackage{[small,compact]{titlesec}

%=========================== 特殊文本元素宏包 ==============================%
\usepackage{nicefrac}                % 在正文文本中排版分式时,可以用它来得到较好的排版效果。
\usepackage{units}                   % 基于 nicefrac 宏包,提供对计量单位比较美观的排版效果。
%\usepackage{titletoc}                % 控制目录的宏包
\usepackage{listings}                % 源代码宏包
\usepackage{makeidx}                 % 建立索引宏包
\makeindex

% ---------生成有书签的pdf及其开关
%\def\a{true}
%\ifx\a\useyap
%\AtBeginDvi{\special{pdf:tounicode GBK-EUC-UCS2}} % GBK -> Unicode
%\fi


%------公式,参考文献的标签显示在页边,在论文修改时可以使用------
%---%\usepackage{labname}               % 使用草稿专用宏包
%\usepackage[amsmath]{labelname}

\usepackage[subnum]{cases}           % 公式环境cases
%\usepackage{cases}
