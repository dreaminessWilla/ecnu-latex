% !Mode:: "TeX:UTF-8"
% UTF-8 编辑器
\chapter{表格样本}
\section{各种表格实例}
\subsection{无线表格,小数点对齐}
\tabbingsep=0pt
\begin{tabbing}
\heiti 行星\hspace{10mm}\=\heiti 赤道半径 km\hspace{10mm}
  \=\heiti 公转周期 d\\
\pushtabs
水星\hspace{20mm} \=  2\'.439\hspace{18mm}
\= 87\'.9\\
金星 \>      6\'.1  \> 224\'.682\\
地球 \> 6378\'.14 \> 365\'.25
\poptabs
\end{tabbing}
\subsection{固定列宽和自动伸缩列宽的三线表}
% 使用宏包booktabs---三线表
%%%%% ===== 固定列宽和自动伸缩列宽表格 ===========================================
% 修改\\, 否则必须用\tabularnewline代替\\
\newcommand{\PreserveBackslash}[1]{\let\temp=\\#1\let\\=\temp}
\newcolumntype{C}[1]{>{\PreserveBackslash\centering}p{#1}}
\newcolumntype{R}[1]{>{\PreserveBackslash\raggedleft}p{#1}}
\newcolumntype{L}[1]{>{\PreserveBackslash\raggedright}p{#1}}
\begin{table}[htbp]
  \centering\caption{\label{table:tab9}2000
           和~2004 年中国制造业产品的出口份额}
  \begin{tabular}{l*{2}{R{2cm}}}
  \toprule
    & 2000 & 2004 \\
  \midrule
    钢铁                  &  3.1 &  5.2 \\
    化学制品              &  2.1 &  2.7 \\
    办公设备及电信设备    &  4.5 & 15.2 \\
    汽车产品              &  0.3 &  0.7 \\
    纺织品                & 10.4 & 17.2 \\
    服装                  & 18.3 & 24.0 \\
  \bottomrule
  \end{tabular}
\end{table}


\subsection{跨页长表格: 使用 \texttt{longtable}宏包}
 \renewcommand{\arraystretch}{0.8}
\begin{longtable}{p{3cm}|p{8cm}}
\caption{统计分析中常用的函数与作用\label{ltab-1}}\\
 \toprule
 \rowcolor[gray]{.9}
统计函数 & 作用\\
 \midrule
$\max(x)$ &  返回向量x中最大的元素\\
$\min(x)$ & 返回向量x中最小的元素\\
$\texttt{which.max}(x)$ & 返回向量x中最大元素的下标\\
$\texttt{which.min}(x)$ & 返回向量x中最小元素的下标\\
$\texttt{mean}(x)$ & 计算样本(向量)x的均值\\
$\texttt{median}(x)$ & 计算样本(向量)x的中位数\\
$\texttt{mad}(x)$ & 计算中位绝对离差\\
$\texttt{var}(x)$ &  计算样本(向量)x的方差\\
$\texttt{sd}(x)$ & 计算向量x的标准差\\
$\texttt{range}(x)$ & 返回长度为2的向量: $\texttt{c}(\min(x),\max(x))$\\
$\texttt{IQR}(x)$ & 计算样本的四分位数极差\\
$\texttt{quantile}(x)$ & 计算样本常用的分位数\\
$\texttt{summary}(x)$ & 计算常用的描述性统计量(最小、最大、平均值、中位数和四分位数)\\
$\cdots$ & $\cdots$\\
  \bottomrule
\end{longtable}


\subsection{单元格合并}
\renewcommand{\multirowsetup}{\centering}
\begin{table}[htbp]
\centering
\caption{单元格合并}\label{merge}
\begin{tabular}{c|r|r}\hline
\multirow{3}[2]{20mm}{行星} & \multicolumn{2}{p{40mm}}%
{\centering 与太阳的距离 \\ (million km)}
\bigstrut[t] \\ \cline{2-3}
& \multicolumn{1}{c}{最远点}
& \multicolumn{1}{c}{最近点}
\bigstrut[t] \\ \hline
天王星 & 3011.0 & 2740.0 \bigstrut[t] \\
金星   & 109.0  & 107.6 \\
地球   & 152.6  & 147.4 \\ \hline
\end{tabular}
\end{table}


\subsection{带注释的表格: 使用threeparttable}

\begin{table}[h]
  \caption{2006-2007$^a$\tnote{}赛季科比布莱恩特投篮命中率的后验特征量
  \label{table:tab8}}
  \centering\small
\begin{threeparttable}[b]
\begin{tabular}{lccc}
\toprule
后验 & & 参数 &\\
\cline{2-4}
特征量 & 对数成败优势比$(\theta)$ & 成败优势比$(o)$ & 成功概率$(\pi)$\\
\midrule
均值 & -0.112 & 0.896 & 0.472\\
中位数 & -0.109 & 0.897 & 0.473\\
标准差 & 0.072 & 0.065 & 0.018\\
$Q_{2.5}$\tnote{b} & -0.261 & 0.770 & 0.435\\
$Q_{97.5}$ & 0.026 & 1.026 & 0.507\\
\bottomrule
\end{tabular}
{\tiny 说明:
\begin{tablenotes}
\item[a] 随机游动算法, 剔除期B=500次迭代; 被保留迭代次数$T'=2000$.
\item[b] $Q_{p}$表示分布的$p$分位数.
\end{tablenotes}}
\end{threeparttable}
\end{table}


%\section{表格自动编号}
%%\iffalse%%%%%%%%%%%%%%%%%%%%%%%%%%%%%%%%%%%%%%%%%%%%%%%%%%%
%
%\alphtab %产生形如 4-a,4-b 的编号
%\begin{table}[htbp]
%\begin{minipage}[htbp]{0.3\linewidth}
%  \centering
%  \caption{表格}\label{chap6:biaoge2}
%  \begin{tabular}{|c|c|c|}
%    \hline
%    % after \\: \hline or \cline{col1-col2} \cline{col3-col4} ...
%    123 & 4 & 5 \\
%    \hline
%    67 & 890 & 13 \\
%    \hline
%  \end{tabular}
%\end{minipage}
%\begin{minipage}[htbp]{0.3\linewidth}
%  \centering
%  \caption{表格}\label{chap6:biaoge3}
%  \begin{tabular}{|c|c|c|}
%    \hline
%    % after \\: \hline or \cline{col1-col2} \cline{col3-col4} ...
%    123 & 4 & 5 \\
%    \hline
%    67 & 890 & 13 \\
%    \hline
%  \end{tabular}
%\end{minipage}
%\begin{minipage}[htbp]{0.3\linewidth}
%    \centering
%  \caption{表格}\label{chap6:biaoge4}
%  \begin{tabular}{|c|c|c|}
%    \hline
%    % after \\: \hline or \cline{col1-col2} \cline{col3-col4} ...
%    123 & 4 & 5 \\
%    \hline
%    67 & 890 & 13 \\
%    \hline
%  \end{tabular}
%\end{minipage}
%\end{table}
%\resettab %恢复形如 4,5 的编号
%
%产生形如 4-a,4-b 的编号.
%
%\newpage
%\begin{table}[h]
%  \centering
%  \caption{表格}\label{chap6:biaoge1}
%  \begin{tabular}{|c|c|c|}
%    \hline
%    % after \\: \hline or \cline{col1-col2} \cline{col3-col4} ...
%    123 & 4 & 5 \\
%    \hline
%    67 & 890 & 13 \\
%    \hline
%  \end{tabular}
%\end{table}
%
%恢复原来的表格编号.


%\fi%%%%%%%%%%%%%%%%%%%%%%%%%%%%%%%%%%%%%%%%%%%%%%%%%%%%%%%
